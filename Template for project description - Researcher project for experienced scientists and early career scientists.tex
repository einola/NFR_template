\documentclass{nfr}

\usepackage{hyperref}
\usepackage{soul}

\title{Project title}

\begin{document}

% Project description template
% Researcher Project for Experienced Scientists/Early Career Scientists

% This template consists of
% 
% \textbf{A:} Overview of the chapters and sections to be used as the
% structure of the project description
% 
% \textbf{B:} General guidance
% 
% \textbf{C:} Guidance on the content in each chapter and section
% 
% \textbf{A: Chapters and sections to be used}

% I used omnidoc to translate the word template to tex. The formatting of the
% appendices is therefore not great, but the contents is there.

\maketitle

\section{Excellence}

\subsection{State of the art, knowledge needs and project objectives}

\subsection{Research questions and hypotheses, theoretical approach and methodology}

\subsection{Novelty and ambition}

\section{Impact}

\subsection{Potential for academic impact of the research project}

\subsection{Potential for societal impact of the research project (optional)}

\subsection{Measures for communication and exploitation}

\section{Implementation}

\subsection{Project manager and project group}

\subsection{Project organisation and management}

\newpage

\appendix

\section{Overview of the chapters and sections to be used as the structure of the project description}

This structure is shown above, in the main document.

\section{General guidance}

The project description will be read by the peer reviewers. Together
with the application form and CVs, the project description will be the
basis for the reviewer\textquotesingle s assessment. The proposed
research should be presented clearly, using language that is also
understandable to individuals with a general scientific understanding of
the field. Please note that the referees in the panel where your
application is reviewed do not necessarily work in precisely the same
area as you.

Complete the chapters and sections in the template, following the order
of the items as given in part A, and delete the guidance (part B and C).
The template is designed to address all the elements of the assessment
criteria. Nevertheless, the applicant is strongly advised to read the
evaluation criteria and the call text carefully.

The project description cannot exceed 11 pages, including the list of
references. It is not possible to upload an attachment that exceeds this
page total. The page format must be A4 with 2 cm margins, single spacing
and Arial, Calibri, Times New Roman or similar 11-point font. You are
permitted to use 9-point font for the list of references and figure
captions. Links that are listed in the project description will not be
included in the assessment.

\section{Guidance on the content in each chapter and section}

\textbf{Project title}

Use same title as in the application form.

\textbf{1. Excellence}

This chapter should provide a description of the planned project, to
enable an assessment of its excellence. This chapter is divided in to
three parts that together will be the basis for two separate marks:\\
"Excellence -- potential for advancing the state-of-the-art" and
"Excellence -- quality of R\&D activities".

\textbf{1.1 State of the art, knowledge needs and project objectives}

\begin{itemize}
\item
  Summarise the state of the art of the research area/field the project
  aims to contribute to and describe the knowledge needs and challenges
  that justify the initiation of the project.
\item
  State the overall project objectives and aims in the context of the
  state of the art and knowledge needs.
\end{itemize}

\textbf{1.2 Research questions and hypotheses, theoretical approach and
methodology}

\begin{itemize}
\item
  Describe in detail the research questions and/or hypotheses.
\item
  Describe \ul{thoroughly} the theoretical approach and/or methodology
  chosen to address the project objectives, research questions and/or
  hypotheses. Use a structure of work packages.
\end{itemize}

\begin{quote}
\emph{NB Provide enough detail to enable reviewers to understand what
you are proposing, how it will be carried out and whether it is
feasible.}
\end{quote}

\begin{itemize}
\item
  Give a brief account of possible risks that might endanger achievement
  of project objectives and describe how to manage these risks.
\item
  If relevant, specify why an interdisciplinary approach has been
  chosen.
\item
  If there are ethical issues to consider, describe how these will be
  dealt with.
\item
  If relevant, describe how gender perspectives are taken into account
  in the research content.
\item
  If relevant, describe how potentially undesirable effects from
  carrying out the project, on human and animal health, climate and the
  environment and society at large, can be avoided.
\item
  If relevant, describe how stakeholder/user knowledge will be used.
\end{itemize}

\textbf{Please note:}

Make sure that the theoretical approach and/or choice of methods is well
accounted for and described in detail, and that it is clear how the
methods are adequate for addressing the research questions, hypotheses,
and project objectives.

See more information on
\href{https://www.forskningsradet.no/en/research-policy-strategy/ethical-standards/}{ethical
standards in research}.

\textbf{1.3 Novelty and ambition}

\begin{itemize}
\item
  Describe the potential for development of new knowledge beyond the
  current state of the art, including significant theoretical,
  methodological, experimental and/or empirical advancements.
\item
  Highlight any particularly novel, original or ambitious aspects of the
  project, e.g. in the objectives, research questions/hypotheses,
  approaches and/or methodology.
\end{itemize}

\textbf{2. Impact}

This chapter should describe the importance of the anticipated results
in terms of the potential academic impact and, optionally, the potential
societal impact of the research. The potential impact can be in the
short or longer term. The chapter should also specify the planned
measures for communication and exploitation of the project results.

\textbf{2.1 Potential for academic impact of the research project}

\begin{itemize}
\item
  Building on the description of project objectives and novelty in
  chapter 1, describe clearly why and how the project outputs may
  address important present and/or future scientific challenges and have
  an impact on the research area/field, if successful.
\item
  Describe how to ensure reproducibility and the potential to reuse the
  project outputs through open science practice such as FAIR data,
  software, models, algorithms etc.
\end{itemize}

\textbf{Please note:}

All applications must include a description of the potential for
academic impact of the project.

The description of the potential impact should be project-specific and
related to the planned research. General elaborations on the benefits of
research in a wider context should be avoided.

\textbf{2.2 Potential for societal impact of the research project
(optional)}

\begin{itemize}
\item
  Building on the description of knowledge needs and challenges in
  section 1.1., describe why and how the project outputs, if successful,
  have the potential to meet the mentioned societal challenge(s).
\item
  Describe how new knowledge and project outputs have the potential to
  address one or more of the UN Sustainable Development Goals.
\end{itemize}

\textbf{Please note:}

The description of potential societal impact will be assessed as
follows:

\begin{itemize}
\item
  The panels will assess potential for societal impact \ul{if} the
  applicant has included a description of this in the project
  description
\end{itemize}

For applications initiated in the context of a specific societal
challenge, you should describe the potential for societal impact. If
relevant for the project, this includes describing how the knowledge and
outputs generated in the project can contribute to solving challenges
and/or shed light on important issues related to one or more of the 17
UN Sustainable Development Goals (SDGs)
(\href{https://www.un.org/sustainabledevelopment/}{Link}).

The description of the potential impact should be project-specific and
related to the planned research. General elaborations on the benefits of
research in a wider context should be avoided.

\textbf{2.3 Measures for communication and exploitation}

\begin{itemize}
\item
  Describe open science practices to ensure early and open sharing and
  wide distribution of research outputs.
\item
  Describe briefly the target audiences, including stakeholders/users,
  of the project outputs (in or beyond the scientific community).
\item
  Outline the scope and plan for dissemination, communication and
  engagement activities.
\item
  Provide a brief description of planned activities that will contribute
  to the realisation of the potential impacts of the project outputs (in
  or beyond the scientific sphere).
\end{itemize}

\textbf{Please note:}

This part of the project description will be the basis for the
assessment of communication and exploitation. Hence, you may leave the
"Communication plan" section in the application form empty.

\textbf{3. Implementation}

This chapter should provide a description of the project team, task
allocation, organisation and management.

\textbf{3.1 Project manager and project group}

\begin{itemize}
\item
  Describe the expertise and experience of the project manager in the
  context of the proposed project to complement the information in the
  CV.
\item
  Describe briefly the project team, including collaborators, to
  complement the information in the CVs. In particular, describe the
  complementarity of the participants in the context of the proposed
  project.
\end{itemize}

\textbf{Please note:}

Avoid repeating information already contained in the CVs. Focus on the
concrete roles and tasks and how the project team, including key
collaborators, is suitable and adequate for the research project.

\textbf{3.2 Project organisation and management}

\begin{itemize}
\item
  Describe the work plan using Gantt chart(s) or other visual
  representations of the plan.
\item
  Describe the allocation of tasks to the project team members, linking
  the tasks to specific work packages.
\item
  Provide a brief overview of research infrastructure and other
  resources that will be essential for carrying out the proposed
  project.
\item
  Describe the organisation and management structure.
\end{itemize}

\textbf{Please note:}

The ambitions of the project, described in chapter 1, should be
realistic in terms of resources such as personnel, expertise, research
infrastructure etc., described in this chapter.

\end{document}
